\documentclass[12pt,a4paper]{article}
\usepackage[utf8]{inputenc}
\usepackage[brazil]{babel}
\usepackage{graphicx}
\usepackage{float}
\usepackage{hyperref}
\usepackage{geometry}
\usepackage{fancyhdr}
\usepackage{titlesec}
\usepackage{xcolor}
\usepackage{enumitem}
\usepackage{setspace}
\usepackage{times}
\onehalfspacing

% Configuração de página
\geometry{
    a4paper,
    left=2.5cm,
    right=2.5cm,
    top=3cm,
    bottom=3cm
}

% Cabeçalho e rodapé
\pagestyle{fancy}
\fancyhf{}
\fancyhead[L]{\textit{Manual de Reserva de Salas de Reunião}}
\fancyhead[R]{\textit{Intranet Grupo Plínio Fleck}}
\fancyfoot[C]{\thepage}

% Cores personalizadas
\definecolor{primary}{RGB}{0, 102, 204}
\definecolor{secondary}{RGB}{102, 102, 102}

% Configuração de títulos
\titleformat{\section}
{\color{primary}\normalfont\Large\bfseries}
{\thesection}{1em}{}

\titleformat{\subsection}
{\color{secondary}\normalfont\large\bfseries}
{\thesubsection}{1em}{}

% Informações do documento
\title{
    \begin{center}
        \vspace*{-3cm}
        {\Large\textbf{GRUPO PLÍNIO FLECK}}\\[1cm]
        {\Huge\textbf{Manual de Treinamento}}\\[0.5cm]
        {\Large Sistema de Reserva de Salas de Reunião}\\[0.3cm]
        {\large Intranet Grupo Plínio Fleck}\\[3cm]
        {\normalsize \textbf{Departamento de Tecnologia da Informação}}\\[0.3cm]
    \end{center}
}

\author{}
\date{}

\begin{document}

\maketitle
\thispagestyle{empty}

\vspace{2cm}

\begin{center}
\textbf{Resumo}
\end{center}

\noindent
Este manual tem como objetivo orientar os colaboradores sobre o uso do \textbf{Sistema de Reserva de Salas de Reunião} disponível na Intranet. O sistema permite visualizar a disponibilidade das salas, realizar agendamentos, gerenciar reservas e cancelar reuniões quando necessário.

\vfill
\begin{center}
    \textit{Campo Bom, \today}
\end{center}
\newpage
\tableofcontents
\newpage

\section{Introdução}

O Sistema de Reserva de Salas de Reunião foi desenvolvido para facilitar o agendamento e gerenciamento de espaços de reunião na empresa. Com ele, é possível:

\begin{itemize}[leftmargin=*]
    \item Visualizar a disponibilidade das salas em tempo real
    \item Agendar reuniões de forma rápida e intuitiva
    \item Convidar participantes que receberão notificações por e-mail
    \item Gerenciar e cancelar suas reservas
    \item Evitar conflitos de horários
    \item Visualizar eventos recorrentes
\end{itemize}

\section{Acessando o Sistema}

Para acessar o Sistema de Reserva de Salas, siga os passos abaixo:

\begin{enumerate}[leftmargin=*]
    \item Acesse a Intranet da empresa através do navegador
    \item No menu principal, clique em \textbf{``Calendário de Salas''}
    \item O sistema será carregado automaticamente
\end{enumerate}

\begin{figure}[H]
    \centering
    \fbox{\includegraphics[width=0.9\textwidth]{print01.png}}
    \caption{Tela inicial da Intranet com acesso ao Calendário de Salas}
    \label{fig:acesso}
\end{figure}

\newpage

\section{Interface do Calendário}

\subsection{Visão Geral}

A interface principal do sistema apresenta um calendário completo onde você pode visualizar todas as reservas das salas de reunião.

\begin{figure}[H]
    \centering
    \fbox{\includegraphics[width=0.9\textwidth]{print02.png}}
    \caption{Visão geral da interface do calendário de salas}
    \label{fig:interface}
\end{figure}

\subsection{Elementos da Interface}

A interface é composta pelos seguintes elementos:

\begin{itemize}[leftmargin=*]
    \item \textbf{Barra de Navegação:} Permite navegar entre os meses (setas esquerda/direita)
    \item \textbf{Botão ``Hoje'':} Retorna rapidamente para o mês atual
    \item \textbf{Indicador do Mês/Ano:} Mostra o período sendo visualizado
    \item \textbf{Botão ``Nova Reunião'':} Abre o formulário para criar um novo agendamento
    \item \textbf{Grade do Calendário:} Exibe os dias do mês com os agendamentos
    \item \textbf{Indicadores de Eventos:} Pequenos marcadores nos dias com reuniões agendadas
\end{itemize}

\newpage

\section{Visualizando Agendamentos}

\subsection{Ver Reuniões de um Dia}

Para visualizar as reuniões agendadas em um dia específico:

\begin{enumerate}[leftmargin=*]
    \item Clique sobre o dia desejado no calendário
    \item Uma janela flutuante será aberta mostrando todos os agendamentos daquele dia
    \item As reuniões são listadas em ordem cronológica
\end{enumerate}

\begin{figure}[H]
    \centering
    \fbox{\includegraphics[width=0.9\textwidth]{print03.png}}
    \caption{Visualização de agendamentos de um dia específico}
    \label{fig:ver_dia}
\end{figure}

\textit{\textbf{Descrição do Print 3:} Capturar a tela com um dia selecionado (deve estar destacado) e o painel lateral aberto mostrando a lista de reuniões daquele dia, com informações como título, horário e sala de cada reunião.}

\subsection{Detalhes de uma Reunião}

Cada reunião exibida mostra as seguintes informações:

\begin{itemize}[leftmargin=*]
    \item Título da reunião
    \item Horário de início e término
    \item Nome da sala reservada
    \item E-mail do organizador
    \item Lista de participantes (se houver)
\end{itemize}

\begin{figure}[H]
    \centering
    \fbox{\includegraphics[width=0.7\textwidth]{print04.png}}
    \caption{Detalhes de uma reunião específica}
    \label{fig:detalhes}
\end{figure}

\textit{\textbf{Descrição do Print 4:} Capturar um card ou modal mostrando os detalhes completos de uma reunião específica, incluindo todos os campos: título, data, horário de início e fim, sala, organizador e lista de participantes.}

\newpage

\section{Criando uma Nova Reserva}

\subsection{Passo 1: Abrindo o Formulário}

Para criar uma nova reserva de sala:

\begin{enumerate}[leftmargin=*]
    \item Clique no botão \textbf{``Nova Reunião''} localizado no topo do calendário
    \item O formulário de agendamento será aberto
\end{enumerate}

\begin{figure}[H]
    \centering
    \fbox{\includegraphics[width=0.9\textwidth]{print05.png}}
    \caption{Botão para criar nova reunião}
    \label{fig:botao_nova}
\end{figure}

\textit{\textbf{Descrição do Print 5:} Capturar a parte superior do calendário com destaque para o botão ``Nova Reunião'' ou ``Agendar Reunião''. O botão deve estar visível e em evidência.}

\subsection{Passo 2: Preenchendo os Dados}

No formulário de agendamento, preencha os seguintes campos:

\begin{itemize}[leftmargin=*]
    \item \textbf{Título da Reunião:} Nome descritivo da reunião
    \item \textbf{Data:} Selecione o dia da reunião usando o calendário
    \item \textbf{Horário de Início:} Hora que a reunião começará
    \item \textbf{Horário de Término:} Hora que a reunião terminará
    \item \textbf{Sala:} Selecione a sala desejada no menu dropdown
    \item \textbf{Participantes:} (Opcional) Liste os e-mails dos participantes separados por vírgula
\end{itemize}

\begin{figure}[H]
    \centering
    \fbox{\includegraphics[width=0.8\textwidth]{print06.png}}
    \caption{Formulário de agendamento de reunião}
    \label{fig:formulario}
\end{figure}

\textit{\textbf{Descrição do Print 6:} Capturar o formulário completo de criação de reunião, mostrando todos os campos vazios ou parcialmente preenchidos como exemplo. Deve incluir todos os campos mencionados acima e os botões de ação (Salvar/Cancelar).}

\newpage

\subsection{Passo 3: Selecionando a Sala}

Ao clicar no campo de seleção de sala:

\begin{enumerate}[leftmargin=*]
    \item Um menu dropdown será exibido com todas as salas disponíveis
    \item As salas são listadas em ordem alfabética
    \item Clique na sala desejada para selecioná-la
\end{enumerate}

\begin{figure}[H]
    \centering
    \fbox{\includegraphics[width=0.7\textwidth]{print07.png}}
    \caption{Menu de seleção de sala de reunião}
    \label{fig:selecao_sala}
\end{figure}

\textit{\textbf{Descrição do Print 7:} Capturar o campo de seleção de sala com o dropdown aberto, mostrando a lista de salas disponíveis. Exemplo: Sala 1, Sala 2, Sala de Conferência, etc.}

\subsection{Passo 4: Adicionando Participantes}

Para adicionar participantes à reunião:

\begin{enumerate}[leftmargin=*]
    \item No campo ``Participantes'', digite os e-mails separados por vírgula
    \item Exemplo: \texttt{usuario1@empresa.com.br, usuario2@empresa.com.br}
    \item Os participantes receberão um e-mail com os detalhes da reunião e arquivo ICS
    \item Este campo é opcional
\end{enumerate}

\begin{figure}[H]
    \centering
    \fbox{\includegraphics[width=0.8\textwidth]{print08.png}}
    \caption{Campo para adicionar participantes}
    \label{fig:participantes}
\end{figure}

\textit{\textbf{Descrição do Print 8:} Capturar o campo de participantes preenchido com alguns exemplos de e-mails separados por vírgula. Mostrar como deve ser o formato correto de entrada.}

\newpage

\subsection{Passo 5: Confirmando o Agendamento}

Após preencher todos os campos obrigatórios:

\begin{enumerate}[leftmargin=*]
    \item Revise todas as informações inseridas
    \item Clique no botão \textbf{``Agendar''} ou \textbf{``Salvar''}
    \item Aguarde a confirmação do sistema
    \item Uma mensagem de sucesso será exibida
    \item O calendário será atualizado automaticamente
\end{enumerate}

\begin{figure}[H]
    \centering
    \fbox{\includegraphics[width=0.7\textwidth]{print09.png}}
    \caption{Mensagem de confirmação de agendamento}
    \label{fig:confirmacao}
\end{figure}

\textit{\textbf{Descrição do Print 9:} Capturar a mensagem de sucesso/confirmação que aparece após criar um agendamento com sucesso. Pode ser um toast, modal ou alerta mostrando ``Reunião agendada com sucesso'' ou mensagem similar.}

\section{Verificando Conflitos de Horário}

O sistema impede automaticamente conflitos de agendamento:

\begin{itemize}[leftmargin=*]
    \item Se tentar agendar uma sala já ocupada no horário selecionado, o sistema exibirá um erro
    \item A mensagem indicará que a sala não está disponível naquele horário
    \item Você deverá escolher outro horário ou outra sala
\end{itemize}

\begin{figure}[H]
    \centering
    \fbox{\includegraphics[width=0.7\textwidth]{print10.png}}
    \caption{Mensagem de erro por conflito de horário}
    \label{fig:conflito}
\end{figure}

\textit{\textbf{Descrição do Print 10:} Capturar a mensagem de erro que aparece quando tenta-se agendar uma sala que já está ocupada no horário selecionado. Deve mostrar claramente a mensagem de erro informando sobre o conflito.}

\newpage

\section{Cancelando uma Reserva}

\subsection{Processo de Cancelamento}

Para cancelar uma reserva existente:

\begin{enumerate}[leftmargin=*]
    \item Localize a reunião no calendário clicando no dia correspondente
    \item Na lista de reuniões, identifique a reunião que deseja cancelar
    \item Clique no botão \textbf{``Cancelar''} ou ícone de cancelamento
    \item O sistema solicitará um código de verificação (OTP)
\end{enumerate}

\begin{figure}[H]
    \centering
    \fbox{\includegraphics[width=0.8\textwidth]{print11.png}}
    \caption{Botão para cancelar reunião}
    \label{fig:botao_cancelar}
\end{figure}

\textit{\textbf{Descrição do Print 11:} Capturar um card de reunião com destaque para o botão ou ícone de cancelamento. Deve mostrar uma reunião específica com a opção de cancelar visível.}

\subsection{Verificação por OTP}

Por segurança, o cancelamento requer verificação:

\begin{enumerate}[leftmargin=*]
    \item Um código OTP (One-Time Password) será enviado para seu e-mail
    \item Acesse seu e-mail e copie o código recebido
    \item Digite o código no campo de verificação
    \item Clique em \textbf{``Confirmar Cancelamento''}
\end{enumerate}

\begin{figure}[H]
    \centering
    \fbox{\includegraphics[width=0.7\textwidth]{print12.png}}
    \caption{Tela de verificação OTP para cancelamento}
    \label{fig:otp}
\end{figure}

\textit{\textbf{Descrição do Print 12:} Capturar a tela ou modal de verificação OTP, mostrando o campo onde o usuário deve inserir o código recebido por e-mail e o botão de confirmação.}

\newpage

\subsection{E-mail com Código OTP}

O e-mail de verificação contém:

\begin{itemize}[leftmargin=*]
    \item Código OTP (geralmente 6 dígitos)
    \item Informações sobre a reunião a ser cancelada
    \item Prazo de validade do código (normalmente 10 minutos)
\end{itemize}

\begin{figure}[H]
    \centering
    \fbox{\includegraphics[width=0.8\textwidth]{print13.png}}
    \caption{E-mail com código OTP para cancelamento}
    \label{fig:email_otp}
\end{figure}

\textit{\textbf{Descrição do Print 13:} Capturar um exemplo de e-mail recebido com o código OTP. Deve mostrar o código em destaque, informações da reunião e instruções para uso do código.}

\subsection{Confirmação de Cancelamento}

Após inserir o código correto:

\begin{enumerate}[leftmargin=*]
    \item O sistema validará o código OTP
    \item A reunião será removida do calendário
    \item Uma mensagem de confirmação será exibida
    \item Todos os participantes receberão um e-mail de cancelamento
\end{enumerate}

\begin{figure}[H]
    \centering
    \fbox{\includegraphics[width=0.7\textwidth]{print14.png}}
    \caption{Confirmação de cancelamento da reunião}
    \label{fig:confirmacao_cancelamento}
\end{figure}

\textit{\textbf{Descrição do Print 14:} Capturar a mensagem de sucesso após cancelamento da reunião, mostrando ``Reunião cancelada com sucesso'' ou mensagem similar.}

\newpage

\section{E-mails e Notificações}

\subsection{E-mail de Convite}

Quando uma reunião é criada, os participantes recebem um e-mail contendo:

\begin{itemize}[leftmargin=*]
    \item Título da reunião
    \item Data e horário
    \item Sala reservada
    \item Nome do organizador
    \item Arquivo ICS anexo (para adicionar ao calendário pessoal)
\end{itemize}

\begin{figure}[H]
    \centering
    \fbox{\includegraphics[width=0.8\textwidth]{print15.png}}
    \caption{E-mail de convite para reunião}
    \label{fig:email_convite}
\end{figure}

\textit{\textbf{Descrição do Print 15:} Capturar um exemplo de e-mail de convite que os participantes recebem ao serem incluídos em uma reunião. Deve mostrar todos os detalhes da reunião e mencionar o anexo ICS.}

\subsection{Anexo ICS}

O arquivo ICS (iCalendar) permite:

\begin{itemize}[leftmargin=*]
    \item Adicionar automaticamente a reunião ao calendário pessoal
    \item Compatível com Outlook, Google Calendar, Apple Calendar, etc.
    \item Basta clicar no arquivo para importar
    \item Inclui lembretes automáticos
\end{itemize}

\begin{figure}[H]
    \centering
    \fbox{\includegraphics[width=0.7\textwidth]{print16.png}}
    \caption{Arquivo ICS anexado ao e-mail}
    \label{fig:ics}
\end{figure}

\textit{\textbf{Descrição do Print 16:} Capturar o anexo ICS no e-mail, mostrando o arquivo anexado (geralmente com ícone de calendário) e seu nome. Pode também mostrar como fica após importar no calendário pessoal.}

\newpage

\section{Recursos Avançados}

\subsection{Eventos Recorrentes}

O sistema suporta visualização de eventos recorrentes:

\begin{itemize}[leftmargin=*]
    \item Reuniões que se repetem periodicamente são exibidas no calendário
    \item Identifique padrões de ocupação das salas
    \item Facilita o planejamento de longo prazo
\end{itemize}

\subsection{Navegação por Período}

Para facilitar a visualização:

\begin{itemize}[leftmargin=*]
    \item Use as setas para navegar entre meses
    \item Clique em ``Hoje'' para retornar ao mês atual
    \item Visualize meses futuros para planejar com antecedência
\end{itemize}

\begin{figure}[H]
    \centering
    \fbox{\includegraphics[width=0.9\textwidth]{print17.png}}
    \caption{Controles de navegação entre meses}
    \label{fig:navegacao}
\end{figure}

\textit{\textbf{Descrição do Print 17:} Capturar os controles de navegação do calendário (setas para anterior/próximo mês e botão ``Hoje'') em destaque. Pode mostrar a transição entre meses diferentes.}

\section{Boas Práticas}

Para melhor aproveitamento do sistema, recomenda-se:

\begin{enumerate}[leftmargin=*]
    \item \textbf{Planeje com antecedência:} Reserve as salas com pelo menos 24h de antecedência
    \item \textbf{Seja pontual:} Respeite os horários de início e fim das reuniões
    \item \textbf{Cancele reservas não utilizadas:} Libere a sala se a reunião for cancelada
    \item \textbf{Títulos descritivos:} Use títulos claros para facilitar a identificação
    \item \textbf{Confirme participantes:} Verifique se todos os e-mails estão corretos
    \item \textbf{Verifique conflitos:} Sempre consulte o calendário antes de agendar
    \item \textbf{Atualize informações:} Mantenha seu e-mail de contato atualizado
\end{enumerate}

\newpage

\section{Solução de Problemas}

\subsection{Problemas Comuns}

\subsubsection{Não consigo agendar uma sala}

\textbf{Possíveis causas:}
\begin{itemize}[leftmargin=*]
    \item A sala já está ocupada no horário desejado
    \item Horário de término anterior ao horário de início
    \item Campos obrigatórios não preenchidos
\end{itemize}

\textbf{Solução:}
\begin{itemize}[leftmargin=*]
    \item Verifique a disponibilidade da sala no calendário
    \item Confira se os horários estão corretos
    \item Preencha todos os campos obrigatórios
\end{itemize}

\subsubsection{Não recebi o código OTP}

\textbf{Possíveis causas:}
\begin{itemize}[leftmargin=*]
    \item E-mail na caixa de spam
    \item E-mail incorreto cadastrado
    \item Atraso no envio do e-mail
\end{itemize}

\textbf{Solução:}
\begin{itemize}[leftmargin=*]
    \item Verifique a caixa de spam/lixo eletrônico
    \item Aguarde alguns minutos e tente novamente
    \item Entre em contato com o suporte de TI
\end{itemize}

\subsubsection{Participantes não receberam o convite}

\textbf{Possíveis causas:}
\begin{itemize}[leftmargin=*]
    \item E-mails digitados incorretamente
    \item E-mails na caixa de spam dos participantes
    \item Problemas no servidor de e-mail
\end{itemize}

\textbf{Solução:}
\begin{itemize}[leftmargin=*]
    \item Verifique se os e-mails foram digitados corretamente
    \item Peça aos participantes para verificarem o spam
    \item Reenvie o convite ou notifique manualmente
\end{itemize}

\newpage

\section{Suporte Técnico}

Em caso de dúvidas ou problemas não solucionados por este manual:

\vspace{0.5cm}

\begin{itemize}[leftmargin=*]
    \item \textbf{Departamento:} TI - Tecnologia da Informação
    \item \textbf{E-mail:} suporte@empresa.com.br
    \item \textbf{Ramal:} XXXX
    \item \textbf{Horário de atendimento:} Segunda a sexta, das 8h às 18h
\end{itemize}

\vspace{1cm}

\section{Controle de Versões}

\begin{table}[H]
\centering
\begin{tabular}{|c|c|p{6cm}|c|}
\hline
\textbf{Versão} & \textbf{Data} & \textbf{Descrição} & \textbf{Autor} \\
\hline
1.0 & \today & Versão inicial do manual & TI \\
\hline
& & & \\
\hline
& & & \\
\hline
\end{tabular}
\caption{Histórico de versões do documento}
\end{table}

\vspace{2cm}

\begin{center}
\hrule
\vspace{0.3cm}
\textit{Este manual é um documento interno da empresa e não deve ser distribuído externamente sem autorização.}
\end{center}

\end{document}
